\documentclass[a4paper,11pt]{report}
\usepackage[swedish]{babel}
\usepackage[utf8]{inputenc}
\usepackage{fancyhdr}
\pagestyle{fancy}
\chead{Uppgiftsbeskrivning Exjobb}
\cfoot{Joel Gärtner}
\begin{document}
\section*{1. Exjobbarens namn och E-post address}
\begin{minipage}[t]{7cm}
    {\textbf{Namn:}}\\
Joel Gärtner

\end{minipage}
\begin{minipage}[t]{7cm}
    {\textbf{E-post address:}}\\
jgartner@kth.se
\end{minipage}
\section*{2. Preliminär titel}
Ökad säkerhet genom slumptestning
\section*{3. Bakgrund/Förutsättning}
Slump (i form av slumpade databitar) är fundamentala byggstenar i all 
kryptografi; alla chiffer och hash-algoritmer måste använda slumpade 
databitar för att kunna ge någon sorts säkerhet. Moderna applikationer 
använder således också en stor andel slumpdata till sessionsidentifierare,
nonce och liknande. \\
\\
Att generera slumpdata är dock långt ifrån trivialt och det är lätt hänt
att det är går att hitta mönster i slumpdatan. Om det går att förutspå 
slumpdatan så försvinner dock av säkerheten man hoppats få genom dem.
Det finns därmed ett säkerhetsintresse att 
kunna förkasta slumpdata som dålig ifall det finns framträdande mönster i det.\\
\\
Tidigare har man utvecklat en mindre kollektion testsviter som testar 
slumpad data (ENT, DieHard, GNU). Dessa verktyg ger en uppskattning av 
entropin på data.  Detta ger en uppskattning av hur pass mycket ren 
slumpdata man faktiskt kan få ut från datakällorna.
Verktygen är dock långt ifrån kompletta och således finns det ett behov av att
kunna erbjuda utvecklare ett ramverk med slumpdata testning som är ”bra nog” 
att avgöra ett datas slumpegenskaper. \\
\\
Den praktiska delen av arbetet ska utföras på företaget Omegapoint.

\section*{4. Vetenskaplig fråga}
Det finns idag flera metoder som kan upptäcka fenomen i data som inte borde 
finnas i äkta slumpdata. Dessa metoder kan användas för att förkasta en del
data som vid första anblick ser slumpmässigt ut, men som vid djupare granskning
visar sig inte bete sig slumpmässigt. Många av dessa metoder är idag 
implementerade i verktyg som ger utvecklare möjligheten att testa sitt 
data och kan visa på potentiella sårbarheter relaterat till dåligt genererad 
slump. 
Nya metoder har dock på senare tid kommit fram för att uppskatta entropin genom
prediktorer. Dessa har lyckats ge en lägre uppskattning på entropin på 
slumpdata som traditionella tester har överskattat entropin på. 
Vidare studier av dessa prediktorer är därmed intressanta.
Frågan är då vad för typ av prediktorer som hittills har implementerats
och hur dom står sig mot traditionella tester i olika sammanhang. 
Frågan är även om det finns fler typer av sådana prediktorer som skulle
kunna implementeras för att ge ännu bättre uppskattning av entropin på olika
typer av slumpkällor. 
\subsection*{Forskningsområde}
Kryptologi, datalogi
\subsection*{Koppling till Forskning / Utveckling}
Vikten av att mata kryptografiska protokoll eller motsvarande med tillräckligt 
bra slumpdata är varken trivialt eller något som praktiseras i större
utsträckning inom systemutveckling idag. Ett systems säkerhet vilar på att 
underliggande kryptografiska funktioner är intakta, viket de potentiellt inte 
är om dålig slump genereras, således behövs det (enkla) metoder att kontrollera 
detta i syfte att dels ge utvecklare ett verktyg som kan eliminera en del
sårbarheter. Användandet av ett sådant verktyg kan även leda till att 
utvecklare blir mer medvetna om denna typ av sårbarheter vilket förhoppningsvis
leder till att dessa i större grad kan förhindras. Aktuell forskning 
tittar på problemet i viss utsträckning men kanske inte tillfredsställande 
mycket i relation till hur det skall användas och tillämpas i praktiken 
(med moderna system). Om nya bättre metoder kan utvecklas kan detta även
ligga till grunden för vidare standardisering av entropi och 
slumpdatatestning såsom gjorts av NIST.  Intresset för frågeställningen 
ligger inom domänen 
för all systemutveckling och systemarkitektur som har krav på säkerhet.
\subsection*{Undersökningsmetod}
Litteraturstudie av vilka teoretiska tester/testmodeller som finns implementerade
idag (och hur) och vilka som inte finns implementerade. Undersökningen består i
att göra ett enkelt proof-of-concept av ett testverktyg med sammanställning av 
väl valda teoretiska tester som bedöms lämpliga och genom att analysera resultaten
efter körning av verktyget se vilken typ av information som potentiellt skulle
kunna ges för att öka säkerheten/identifiera sårbarheter.
\subsection*{Hypotes}
\begin{itemize}
\item Funna metoder (kombination av metoder) ger signifikanta 
    indikationer på att ett system genererar dålig slumpdata.
\item Funna metoder är minst lika bra, men kanske bättre än 
redan befintliga kombinationer av metoder, vilket oavsett 
kompletterar området ''testning av kryptografiskt säker slumpdata''. 
\item Funna metoder implementerade som ett verktyg som är lätt att 
    använda leder till att systemutvecklare kan börja använda det
    och således minska risken att dålig slumpdata används.
\end{itemize}
\subsection*{Utvärdering}
\begin{itemize}
    \item Provköra verktyget och jämföra med andra befintliga tester, 
    \item Provköra verktyget skarpt och låta systemutvecklare bedöma graden
        av ökad säkerhet som en direkt implikation av ett enkelt,
        användarvänligt men teoretiskt starkt verktyg.
\end{itemize}

\section*{5. Exjobbarens bakgrund}

Kommer från Teknisk Fysik med mycket matematik vilket kan komma till användning.
Fortsatt med masterinriktning datalogi vilket med spår inom datasäkerhet vilket
lämpar sig väl för exjobbet.  Har även som del av utbildningen läst kurs inom
kryptologi vilket ligger till fokus för exjobbet.

\section*{6. Handledare på företaget}
Hannes Salin, säkerhetskonsult, handledare från företagets sida.
\section*{7. Avgränsning/Resurser}
Påbörjad undersökning och litteraturstudie finns, identifierade användningsområden finns att pröva detta på
\section*{8. Behörighet och Studieplanering}

\subsection*{Behörighet}
Har läst mer än tillräckligt kurser under tre första åren på teknisk fysik
och har fått alla, inklusive kandidat examensarbetet godkända. Har även
läst gott om Master-nivå kurser och det är mer än 60 hp på avancerad nivå
som är helt avklarade. Klar med alla obligatoriska kurser förutom 
programsammanhållande kursen. Även klar med vetenskapsteori kursen DA2210.
\subsection*{Studieplanering}
De ända kurser som återstår för examen är programsammanhållande kursen,
examensarbetet samt någon kurs inom spåret datasäkerhet på åtminstone 3hp. 
Kursen inom spåret läser jag för tillfället och bör vara klar med till jul
och examensarbete och programsammanhållande kursen bör båda vara klara
i slutet av period 4. 

\end{document}

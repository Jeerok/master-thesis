\documentclass[a4paper,11pt]{report}
\usepackage[swedish]{babel}
\usepackage{fancyhdr}
\usepackage{cite}
\usepackage{amsmath}
\usepackage[utf8]{inputenc}

\pagestyle{fancy}
\lhead{Joel Gärtner}
\chead{\large\textbf{Project specification}}
\rhead{jgartner@kth.se}
\begin{document}
\section*{Formalities}
\begin{center}
\textbf{Preliminary Title:}\\ Improved Security Through Randomness Testing
\end{center}

\begin{minipage}[t]{7cm}
\textbf{Name:}\\ Joel Gärtner\\
\textbf{Email:}\\ jgartner@kth.se\\
\textbf{Date:}\\ \today
\end{minipage}
\begin{minipage}[t]{7cm}
\textbf{CSC Supervisor:}\\ Douglas Wikström \\
\textbf{Principal:}\\
Omegapoint \\
\textbf{Principal Supervisor:}\\
 Hannes Salin\\
\end{minipage}
\section*{Background and Objective}
Several cryptographic protocols require some input to be unpredictable 
in order to provide security. Furthermore, any keys used in the protocols 
must also be generated in an unpredictable manner since these are crucial for 
the level of security, thus it must be hard for any adversary to guess or 
compute such keys. In order to get a source of unpredictability,
random generators are used.  These generators can potentially 
be truly random and generate numbers based on some phenomenon which is deemed 
unpredictable. The speed of these generators is however often limited and because
of this it is more common that the generators are pseudo random, i.e.
based upon deterministic algorithms which given a seed for the generator,
produces an arbitrarily long sequence which looks random.
Depending on the application different types of pseudo random generators
are needed. For simulations and similar the most important properties 
is that it is fast to generate numbers and that they behave statistically
in a way that seems random. For cryptographic purposes the generators 
also have to be unpredictable in the meaning that an adversary who sees part
of the generated sequence won't be able to efficiently guess the next bit 
of the sequence with a probability significantly higher than if the sequence 
was truly random. \\

\noindent
These cryptographically secure pseudo random generators are thus such that 
they expand a seed into a sequence which cannot be efficiently distinguished 
from a truly random sequence without knowledge of the seed. If the seed is 
known however, the whole sequence is easily reproducible as the algorithm 
generating the sequence is deterministic. As such that the pseudo random
generator is cryptographically secure will not provide any security if the seed
is predictable. Therefore it is of great importance for a process to select a 
seed of unpredictable bits which can then be expanded into arbitrarily much 
seemingly random data using a cryptographically secure pseudo random generator.
In order to select
this seed there is a requirement for a source of unpredictability. This can
for example be an actual true random generator or another process on the 
generating machine which behaves in an unpredictable manner. It is however not
obvious how much data is needed from the source of unpredictability before 
enough unpredictability has been gathered to the seed. A measure of this 
unpredictability is the entropy of the sample received from the source.
Entropy is a concept taken from thermodynamics which was first used in 
information theory by Shannon with Shannon entropy
\cite{Shannon:2001:MTC:584091.584093}.
A sequence of bits from a source will have an entropy of $1$ per bit if
it is completely random and an entropy of $0$ per bit if it is 
completely predictable. It is easy to estimate the entropy of 
data if the data consists of samples with $n$ bit where each such sample is
generated completely independent from any of the other samples. If this is the 
case then the unpredictability and thus the entropy only depends on the 
probability distributions of the single samples $x$ and thus it is sufficient
to estimate this distribution to get an estimation of the entropy.
In case the samples are not independent identically distributed then it 
is a lot harder to estimate the entropy. \\

\noindent
Several test suites exists which try to determine when data does not behave 
randomly\cite{Ecuyer2007}\cite{Bassham:2010:SRS:2206233}
\cite{brown2017dieharder}. These suites can to some extent 
identify sequences where the samples do not behave like identically distributed
random variables. Most such tests do however only identify when there is 
incorrect behaviour in the sequence and does not give an estimate of the actual
unpredictability of the sequence. As the sequence can somehow be distinguished 
from truly random data, it could be a potential weakness if used in 
cryptographic applications by itself. The sequence does however probably still
contain some amount of unpredictability and this could potentially be extracted
by transforming the sequence so that it statistically behaves more like random
data. This can be done with multiple sources of entropy which can be combined 
in order to create a source of random data which behaves randomly and is 
unpredictable. This source could then be used to get a seed for a
cryptographically secure random generator which could expand this source of 
unpredictable into as much random data as necessary.\\

\noindent
An estimate of the entropy of data is thus useful in order to determine
how much data is necessary to get enough unpredictability for your random
generators. A recent approach to estimating the entropy of data is to use 
predictors\cite{eprint-2015-26658}.
These estimators take the approach of trying to predict the data
which is fed into them as good as possible. The entropy of the data can then
be approximated via the success rate of the predictors. Several predictors were
constructed in order to give good predictions and thus good estimations of the 
entropy. All entropy estimations will however have the problem that they only
estimate the entropy of some specific types of distributions of data while 
other distributions will get wrong estimations. The constructed predictors were 
meant to be general and work in general but more specific estimators will work
better for most types of entropy sources. As such there is the possibility of
a lot more entropy sources which better models more types of data sources. \\

\noindent
A specific application where there is a need for an entropy estimator is for
the random number generator /dev/random on Linux systems. Here data is 
gathered in a pool and the amount of entropy in the data is estimated 
and added to an estimate of the entropy in the pool\cite{/dev/random}.
The random number generator then only returns a number when it deems there 
to be enough entropy in the pool.
Other random number generators which depend on outside entropy 
have similar systems which they have in order to avoid giving numbers to 
the users of the generators when the numbers are deemed too predictable.
The estimates of the available entropy is interesting to analyse as a 
too high estimate of the entropy could be a security risk. There have been 
several examples of security vulnerabilities which were the result of a lack
of entropy when generating random numbers\cite{goldberg1996randomness}
\cite{yilek2009private}.
\section*{Research Questions and Method}
\subsection*{Question}
Can new types of entropy estimators give better estimates than existing
ones on given types of data.
\subsection*{Problem Definition}
Investigating existing entropy sources such as /dev/random and seeing how
they get their entropy and how they estimate the amount they have. Compare the
way these estimates are performed with existing general tests which are meant 
to estimate entropy of general data. Then produce estimators which try to give
better entropy estimates given the data source.
Further comparison of these produced estimators on data which is produced
to follow specific distributions will also be performed in order to get 
values to compare the results to.
\subsection*{Examination Method}
Existing algorithms for entropy estimation to be tested includes the ones
implemented by NIST special publication 800-90b\cite{800-90B} and the ones
in Predictive models for Min Entropy\cite{eprint-2015-26658} as well as 
the estimates used in entropy sources such as /dev/random. These will be 
compared with the developed algorithms, which will include algorithms which
estimate the entropy in the way done by predictors as well as potentially
other type of estimators. \\

\noindent
Preliminary the data to be tested is output from different entropy sources,
before they are mixed in an irreversible way with other data. Furthermore
tests will be performed on data which have been generated to have a specific
distribution and thus have a known entropy, given that the random number 
generator used for the data is a good one. Potentially some study could be
performed on estimating the entropy of random number generators, although only
on weak such, which are not meant to be cryptographically secure.
\subsection*{Expected Scientific Results}
The hypothesis is that new entropy estimators may complement existing 
estimators by providing better estimates for at least some type of data.
This is being tested by producing random data which have a known entropy
with varying distributions and running the estimators against this data. 
The results will hopefully be useful for better estimations of entropy for 
some entropy sources which given better entropy estimations can be either safer
to use or alternatively more efficient while providing the same level of 
security.

\section*{Evaluation and News Value}
\subsection*{Evaluation}
The produced estimators will be compared with existing entropy tests on
different types of data. In order to evaluate the produced estimators the tests
on data which follows known distributions are probably of the most interest as
these have a known correct value on the entropy. This gives a clear indication
on when the newly created estimators produce better values than existing tests.
Any large deviations from the results of existing estimators with the new ones
on real world data may then potentially give information about the properties
of the real world data or alternatively that the produced estimators are not 
good matches for this type of data. 

\subsection*{News Value}
New estimators could be useful for producer of new entropy sources in order to
give better estimates of the produced entropy. Existing entropy sources could
also potentially benefit from new entropy estimators to give a better estimate
and thus potentially eliminate weaknesses existing or give a more efficient 
stream of random data.

\section*{Pilot Study}
Some study of the implementations of pseudo random number generators and
cryptographically secure pseudo random number generators to get a better
understanding of the context. Furthermore study of the Yarrow and
Fortuna algorithms and their implementations may be of interest as these 
use entropy accumulators as integral parts of their algorithm to secure the 
algorithm even in cases where part of the state becomes known to an attacker.
\\

\noindent
There has also been guidelines produced by NIST related to random number
generators and entropy sources. These documents NIST SP 800-90(A,B,C) 
\cite{800-90A}\cite{800-90B}\cite{800-90C} are guidelines related to 
how secure random generators should perform and how sources of entropy 
sources should behave and how they can be tested.
Furthermore researchers at NIST have also produced 
the paper \textit{Predictive Models for Min-entropy Estimation}
\cite{eprint-2015-26658}
related to using predictors to producing better estimates for entropy.
Other sources related to how entropy and randomness testing is performed is
also of interest, including for example how suites such as TestU01 evaluates
data.
\\

\noindent
Furthermore it will need to be investigated what sources of entropy exist today
and how they work. This includes built in entropy generators in operating 
systems but also specific generators built for that specific purpose. Sites
such as \textit{random.org}\cite{haahr2010random} which provide a service which 
provide random data as a service will also be of interest.
\\

\noindent
It may also be useful to study the different cryptographic protocols 
where randomness is useful in order to see what kind of impact too low entropy 
may have in relation to security. In combination with this it will also be 
of interest to investigate the weaknesses which have been discovered that were 
the result of lacking entropy when generating random numbers for cryptographic 
protocols.

\section*{Conditions and Schedule}
The plan is to work more or less full time during the entire thesis project
beginning 17 January until it is finished.
\subsection*{Resources}
There should not be a big demand on resources or equipment.
Potentially output from a hardware source of entropy could be of interest
as well, as it could be another source of entropy to analyse.
\subsection*{Limitations}
The project will deal mainly with entropy estimation. There will thus not be a
lot of focus on what is done with the entropy or how to produce entropy. How 
entropy is produced in current implementations is however of interest as this 
may give insights into how to better estimate the produced entropy. The project
will also not try to estimate the entropy output of cryptographically secure 
random number generators other than as a potential sanity check as these should
be approximately full entropy sources to any efficient and sane estimator.
\subsection*{Collaboration with the principal}
The principal will be available for discussion regarding content and direction
of the project. He will also be able to proof read any material and give 
response on what is written. In case of any needed resources he will also be 
able to give some help in acquiring these.
\subsection*{Schedule}
\begin{description}
    \item [Specification] \hfill
        \begin{description}
            \item[Begin] 17-January 2017
            \item[End] Week 5
            \item[Comment]\hfill \\
                This specification will hopefully be done and approved by 
                week 5.
        \end{description}

    \item[Pilot Study] \hfill
        \begin{description}
            \item[Begin] 17-January 2017
            \item[End] Week 9
            \item[Comment]\hfill \\
                The work with the pilot study was initiated during the work on 
                the specification and will continue when the specification is 
                done for approximately 4 more weeks.
        \end{description}
    \item[Implementation] \hfill
        \begin{description}
            \item[Begin] Week 8
            \item[Working Prototype] Week 12
            \item[End] Week 15
            \item[Comment]\hfill \\
                A working prototype which is relatively close to the final 
                implementation is planned to be finished by Week 12. After this
                testing and fine tuning of the prototype may be performed in 
                order to get good data for the report which will be written 
                in parallel.
        \end{description}
    \item[Report] \hfill
        \begin{description}
            \item[Begin] Week 13
            \item[Finished first draft] Week 17
            \item[End] Week 23
            \item[Comment]\hfill \\
                There will be work on the report during the other parts of the
                thesis, but during the indicated time the main focus will be 
                on the report. A first draft of the report should be finished
                early to give plenty of time for adjustments and comments.
        \end{description}
    \item[Presentation] \hfill
        \begin{description}
            \item[Begin] Week 16
            \item[Finished first draft] Week 18
            \item[End] Week 21
            \item[Comment] The work on the presentation should be done in
                parallel with the fine tuning of the report after the first 
                draft has been completed. A finished first draft of the 
                presentation should be done relatively early to give plenty
                of time to practice the presentation and fine tune it.
        \end{description}
\end{description}

\bibliography{references}{}
\bibliographystyle{plain}
\end{document}

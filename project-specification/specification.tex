\documentclass[a4paper,11pt]{report}
\usepackage[swedish]{babel}
\usepackage{fancyhdr}
\usepackage{amsmath}
\usepackage[utf8]{inputenc}

\pagestyle{fancy}
\lhead{Joel Gärtner}
\chead{\large\textbf{Project specification}}
\rhead{jgartner@kth.se}
\begin{document}
\section*{Formalities}
\begin{center}
\textbf{Preliminary Title:}\\ Improved Security Through Randomness Testing
\end{center}

\begin{minipage}[t]{7cm}
\textbf{Name:}\\ Joel Gärtner\\
\textbf{Email:}\\ jgartner@kth.se\\
\textbf{Date:}\\ \today
\end{minipage}
\begin{minipage}[t]{7cm}
\textbf{CSC Supervisor:}\\ Douglas Wikström \\
\textbf{Principal:}\\
Omegapoint \\
\textbf{Principal Supervisor:}\\
 Hannes Salin\\
\end{minipage}
\section*{Background and Objective}
Several cryptographic protocols require some input to be unpredictable 
in order to function in a way which provides security. Furthermore the 
keys which are in the protocols must also be generated in an unpredictable
manner in order for them to be secure. In order to get a source of 
unpredictability, random generators are used. These generators can potentially 
be truly random and generate numbers based on some phenomenon which is deemed 
unpredictable. The speed of these generators is however often limited and because
of this it is more common that the generators are pseudo random 
based upon deterministic algorithms which given an input, 
a seed for the generator, produces an arbitrarily long sequence which 
looks random. Depending on the application different types of pseudo 
random generators. For simulations and similar the most important properties 
is that it is fast to generate numbers and that they behave statistically
in a way that seems random. For cryptographic purposes the generators 
also have to be unpredictable in the meaning that an adversary who sees part
of the generated sequence won't be able to efficiently guess the next bit 
of the sequence with a probability significantly higher than if the sequence 
was truly random. \\

\noindent
These cryptographically secure pseudo random generators are thus such that 
they expand a seed into a sequence which can't be efficiently distinguished 
from a truly random sequence without knowledge of the seed. If the seed is 
known however, the whole sequence is easily reproducible as the algorithm 
generating the sequence is deterministic. As such the that the pseudo random
generator is cryptographically secure will not provide any security if the seed
is predictable. There is thus a need for a process to unpredictably select a 
seed which can then be expanded into arbitrarily much seemingly random data
using a cryptographically secure pseudo random generator. In order to select
this seed there is a requirement for a source of unpredictability. This can
for example be an actual true random generator or another process on the 
generating machine which behaves in an unpredictable manner. It is however not
obvious how much data is needed from the source of unpredictability before 
enough unpredictability has been gathered to the seed. A measure of this 
unpredictability is the entropy of the sample received from the source.
A sequence of bits from a source will have an entropy of $1$ per bit if
it is completely random and an entropy of $0$ per bit if it is 
completely predictable. It is easy to estimate the entropy of 
data if the data consists of samples with $n$ bit where each such sample is
generated completely independent from any of the other samples. If this is the 
case then the unpredictability and thus the entropy only depends on the 
probabilities of the samples $x$ and can be calculated by estimating the
probabilities of the sample $x$ with the relative frequency of $x$. 
In case the samples are not independent identically distributed then it is a 
lot harder to estimate the entropy. \\

\noindent
Several test suites exists which try to determine when data does not behave 
randomly. These suites can to some extent 
identify sequences where the samples do not behave like identically distributed
random variables. Most such tests do however only identify when there is 
incorrect behaviour in the sequence and does not give an estimate of the actual
unpredictability of the sequence. As the sequence can somehow be distinguished 
from truly random data, it could be a potential weakness if used in 
cryptographic applications by itself. The sequence does however probably still
contain some amount of unpredictability and this could potentially be extracted
by transforming the sequence so that it statistically behaves more like random
data. This can be done with multiple sources of entropy which can be combined 
in order to create a source of random data which behaves randomly and is 
unpredictable. This source could then be used to get a seed for a
cryptographically secure random generator which could expand this source of 
unpredictable into as much random data as necessary.\\

\noindent
An estimate of the entropy of data is thus useful in order to determine
how much data is necessary to get enough unpredictability for your random
generators. A recent approach to estimating the entropy of data is to use 
predictors. These estimators take the approach of trying to predict the data
which is fed into them as good as possible. The entropy of the data can then
be approximated via the success rate of the predictors. Several predictors were
constructed in order to give good predictions and thus good estimations of the 
entropy. All entropy estimations will however have the problem that they only
estimate the entropy of some specific types of distributions of data while 
other distributions will get wrong estimations. The constructed predictors were 
meant to be general and work in general but more specific estimators will work
better for most types of entropy sources. As such there is the possibility of
a lot more entropy sources which better models more types of data sources.

\section*{Research Questions and Method}
\section*{Evaluation and News Value}
\section*{Pilot Study}

\section*{Conditions and Schedule}
\end{document}
